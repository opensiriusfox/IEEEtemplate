% biography section
% 
% If you have an EPS/PDF photo (graphicx package needed) extra braces are
% needed around the contents of the optional argument to biography to prevent
% the LaTeX parser from getting confused when it sees the complicated
% \includegraphics command within an optional argument. (You could create
% your own custom macro containing the \includegraphics command to make things
% simpler here.)
%\begin{IEEEbiography}[{\includegraphics[width=1in,height=1.25in,clip,keepaspectratio]{mshell}}]{Michael Shell}
% or if you just want to reserve a space for a photo:
%\newpage
\vspace{-0.5 cm}

\begin{IEEEbiography}[{\includegraphics[width=1in,height=1.25in,clip,keepaspectratio]{photos/renaud.jpg}}]{Luke Renaud}
%\begin{IEEEbiographynophoto}{Luke Renaud}
(S’13) received the B.S. in Electrical Engineering \textit{summa cumme laude} with a from Washington State University in Pullman, WA in 2013. He is currently pursuing a Ph.D. in RF Microelectronics from Washington State University, Pullman, WA.

If you're reading this paragraph as a reviewer, then the person who submit the paper has forgotten to remove my template BIO from their document. You really should reject their paper, that's pretty sloppy.
%\end{IEEEbiographynophoto}
\vspace{-0.8 cm}
\end{IEEEbiography}

%\vfill

%\begin{IEEEbiography}[{\includegraphics[width=1in,height=1.25in,clip,keepaspectratio]{photos/ylabmate}}]{Your Labmate}
\begin{IEEEbiographynophoto}{Your Labmate}
Bio paragraph
\end{IEEEbiographynophoto}
%\end{IEEEbiography}
%\vspace{-0.8 cm}

%\begin{IEEEbiography}[{\includegraphics[width=1in,height=1.25in,clip,keepaspectratio]{photos/yprof}}]{Your Professor}
\begin{IEEEbiographynophoto}{Your Professor}
Bio paragraph
\end{IEEEbiographynophoto}
%\end{IEEEbiography}

% insert where needed to balance the two columns on the last page with
% biographies
%\newpage

%\begin{IEEEbiographynophoto}{Jane Doe}
%Biography text here.
%\end{IEEEbiographynophoto}

% You can push biographies down or up by placing
% a \vfill before or after them. The appropriate
% use of \vfill depends on what kind of text is
% on the last page and whether or not the columns
% are being equalized.

\vfill

% Can be used to pull up biographies so that the bottom of the last one
% is flush with the other column.
%\enlargethispage{-5in}
